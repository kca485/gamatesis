\chapter{BAB}
\lipsum[3]
\section{Subbab}
\lipsum[5]
\subsection{Anak subbab}
\lipsum[6]
\begin{quote}
\lipsum[1]
\end{quote}
\lipsum[2]
\subsubsection{Subanak subbab}
\lipsum[7]

%%%%%%%%%%%%%%%%%%%%%%%%%%%%%%%%

\chapter{CONTOH TABEL}
Tabel \ref{table:1} adalah contoh.
\begin{table}[h]
\centering
\begin{tabular}{||c c c c||} 
 \hline
 Col1 & Col2 & Col2 & Col3 \\ [0.5ex] 
 \hline\hline
 1 & 6 & 87837 & 787 \\ 
 2 & 7 & 78 & 5415 \\
 3 & 545 & 778 & 7507 \\
 4 & 545 & 18744 & 7560 \\
 5 & 88 & 788 & 6344 \\ [1ex] 
 \hline
\end{tabular}
\caption{Contoh tabel}
\label{table:1}
\end{table}

\section{Contoh Tabel Lagi}
Tabel \ref{table:2} adalah contoh.
\begin{table}[h!]
\centering
\begin{tabular}{||c c c c||} 
 \hline
 Col1 & Col2 & Col2 & Col3 \\ [0.5ex] 
 \hline\hline
 1 & 6 & 87837 & 787 \\ 
 2 & 7 & 78 & 5415 \\
 3 & 545 & 778 & 7507 \\
 4 & 545 & 18744 & 7560 \\
 5 & 88 & 788 & 6344 \\ [1ex] 
 \hline
\end{tabular}
\caption{Contoh tabel}
\label{table:2}
\end{table}

