\documentclass{gamatesis}
\usepackage{fontspec}
\setmainfont[Ligatures=TeX]{Linux Libertine O}
\usepackage{lipsum}


%%%%%%%%%%%%%%%%%%%%%%%%%%%%%%%%%%%%%%%%%%%%%%%%%%%
%            Info pelajar dan instansi            %
%%%%%%%%%%%%%%%%%%%%%%%%%%%%%%%%%%%%%%%%%%%%%%%%%%%

\judul{Contoh Judul Tesis Berukuran 14pt dan Bercetak Tebal yang Sengaja Dipanjangi}

\subjudul{Subjudul Berukuran 12pt}

\karya{TESIS}

\maksud{untuk memenuhi sebagian persyaratan\\ mencapai derajat sarjana S-2}

\prodi{Program Studi Abcdefghij}

\peminatan{Abcde Fghij}

\lambang{lambang.png}

\fillera{diajukan oleh}

\nama{Nama Lengkap}

\nim{01/23456/ABC/7890}

\fillerb{kepada}

\bagiansekolah{Sekolah Pascasarjana}

\fakultas{Fakultas Abcde Fghij}

\departemen{Departemen Abcde}

\universitas{Universitas Abcdefghij}

\lokasi{Abcde}

\tahun{2021}



%%%%%%%%%%%%%%%%%%%%%%%%%%%%%%%%%%%%%%%%%%
%            Struktur tulisan            %
%%%%%%%%%%%%%%%%%%%%%%%%%%%%%%%%%%%%%%%%%%

\begin{document}

\halamansampul

\pagenumbering{roman}

\halamanjudul

\begingroup
\addcontentsline{toc}{chapter}{Lembar Pengesahan}
\halamankosong
\endgroup


\include{pernyataan}

\include{prakata}

\daftarisi

\daftartabel

\daftargambar

\daftarlampiran

\daftar{Contoh1}

\include{abstrak}

\intisari[english]{

\lipsum[1]

\lipsum[2]

}


\pagenumbering{arabic}

\chapter{BAB}
\lipsum[3]
\section{Subbab}
\lipsum[5]
\subsection{Anak subbab}
\lipsum[6]
\subsubsection{Subanak subbab}
\lipsum[7]

%%%%%%%%%%%%%%%%%%%%%%%%%%%%%%%%

\chapter{CONTOH TABEL}
Tabel \ref{table:1} adalah contoh.
\begin{table}[h]
\centering
\begin{tabular}{||c c c c||} 
 \hline
 Col1 & Col2 & Col2 & Col3 \\ [0.5ex] 
 \hline\hline
 1 & 6 & 87837 & 787 \\ 
 2 & 7 & 78 & 5415 \\
 3 & 545 & 778 & 7507 \\
 4 & 545 & 18744 & 7560 \\
 5 & 88 & 788 & 6344 \\ [1ex] 
 \hline
\end{tabular}
\caption{Contoh tabel}
\label{table:1}
\end{table}

\section{Contoh Tabel Lagi}
Tabel \ref{table:2} adalah contoh.
\begin{table}[h!]
\centering
\begin{tabular}{||c c c c||} 
 \hline
 Col1 & Col2 & Col2 & Col3 \\ [0.5ex] 
 \hline\hline
 1 & 6 & 87837 & 787 \\ 
 2 & 7 & 78 & 5415 \\
 3 & 545 & 778 & 7507 \\
 4 & 545 & 18744 & 7560 \\
 5 & 88 & 788 & 6344 \\ [1ex] 
 \hline
\end{tabular}
\caption{Contoh tabel}
\label{table:2}
\end{table}



\lampiran{Lampiran 1}{

Contoh lampiran

\lipsum[9]

\lipsum[10]

}

\pokok{pokok 1}{ tes pokok }


\end{document}
